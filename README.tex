\PassOptionsToPackage{unicode=true}{hyperref} % options for packages loaded elsewhere
\PassOptionsToPackage{hyphens}{url}
%
\documentclass[]{article}
\usepackage{lmodern}
\usepackage{amssymb,amsmath}
\usepackage{ifxetex,ifluatex}
\usepackage{fixltx2e} % provides \textsubscript
\ifnum 0\ifxetex 1\fi\ifluatex 1\fi=0 % if pdftex
  \usepackage[T1]{fontenc}
  \usepackage[utf8]{inputenc}
  \usepackage{textcomp} % provides euro and other symbols
\else % if luatex or xelatex
  \usepackage{unicode-math}
  \defaultfontfeatures{Ligatures=TeX,Scale=MatchLowercase}
\fi
% use upquote if available, for straight quotes in verbatim environments
\IfFileExists{upquote.sty}{\usepackage{upquote}}{}
% use microtype if available
\IfFileExists{microtype.sty}{%
\usepackage[]{microtype}
\UseMicrotypeSet[protrusion]{basicmath} % disable protrusion for tt fonts
}{}
\IfFileExists{parskip.sty}{%
\usepackage{parskip}
}{% else
\setlength{\parindent}{0pt}
\setlength{\parskip}{6pt plus 2pt minus 1pt}
}
\usepackage{hyperref}
\hypersetup{
            pdfborder={0 0 0},
            breaklinks=true}
\urlstyle{same}  % don't use monospace font for urls
\usepackage{longtable,booktabs}
% Fix footnotes in tables (requires footnote package)
\IfFileExists{footnote.sty}{\usepackage{footnote}\makesavenoteenv{longtable}}{}
\setlength{\emergencystretch}{3em}  % prevent overfull lines
\providecommand{\tightlist}{%
  \setlength{\itemsep}{0pt}\setlength{\parskip}{0pt}}
\setcounter{secnumdepth}{0}
% Redefines (sub)paragraphs to behave more like sections
\ifx\paragraph\undefined\else
\let\oldparagraph\paragraph
\renewcommand{\paragraph}[1]{\oldparagraph{#1}\mbox{}}
\fi
\ifx\subparagraph\undefined\else
\let\oldsubparagraph\subparagraph
\renewcommand{\subparagraph}[1]{\oldsubparagraph{#1}\mbox{}}
\fi

% set default figure placement to htbp
\makeatletter
\def\fps@figure{htbp}
\makeatother


\date{}

\begin{document}

Motivation

After witnessing the attempted hostile ideological takeover of a
prominent global non-profit, I became interested in understanding how an
organization might become ideologically homogenous, and, more
importantly, how it might become overrun by zealots. This project is an
attempt to model the ideological population dynamics of an organization
in which individual are interacting, resigning, being fired, and being
hired. The model I've developed is an extension of the model proposed by
applied mathematician Steven Strogatz in his paper ``Encouraging
Moderation.'' The paper is included in the ``Reference Papers'' folder.
Although my model follows the same paradigm as Strogatz's, it does
involve novel mathematical contributions such as, for example, a
polarization metric that construes polarization as how ``separated'' the
subpopulations (defined by worldview) comprising the organization are.

The applications of this model, I realize now, are quite impactful.
Increasingly, certain sectors such as tech, non-profit, and higher
education, are becoming ideologically homogenous. However, ideological
homogeneity is highly undesirable in an organization. Ideally, an
organization will have a diversity of well-informed opinions so that new
challenges can be approached from different perspectives that ``check''
the biases of each other, allowing convergence upon the best or
least-bad solution. Formally, this type of intellectual culture is known
as ``institutionalized disconfirmation.'' It is precisely the norm that
has allows good quality research to flourish in institutions of higher
education and it is also the norm that is most under threat.
Interestingly, the ideological population dynamics of entire industries,
which are simply densely connected collections of organizations, can be
simulated using the modelling framework I propose here.

Further, if one expands one's view of an ``organization'' to a
democratic nation, it is possible (with some modification) to model the
effects of mass disinformation campaigns that flood the virtual public
square with large numbers of ``zealots'' posing as citizens.

The main thrust of this research is to uncover effective strategies for
counteracting the general tendency towards polarization --
conceptualized here as ``component separation'', a recapitulation of
entropy -- and protecting a culture of institutionalized
disconfirmation.

File Descriptions

The files/folders in this repository are as follows:

\begin{enumerate}
\def\labelenumi{\arabic{enumi}.}
\tightlist
\item
  \textbf{Organization\_model.py}: This is contains the two major
  classes that comprise the model, namely, ``Individual'' and
  ``Organization'', which is composed of individuals.
\item
  \textbf{run\_experiments.py}: This is the command line interface that
  allows one to easily run simulations that test various parameters.
\item
  \textbf{Characteristic\_Eqs\_Solver.m}: This is a MatLab script meant
  to solve, for various initial conditions, the system of differential
  equations that fully characterizes the ideological population dynamics
  of an organization.
\item
  \textbf{``Mathematical Characterization of System''}: This is a PDF
  that contains a more rigorous mathematical description of the model.
  This document is currently in progress.
\item
  \textbf{Reference Papers}: This folder contains academic literature
  that has informed this research.
\item
  \textbf{Resultes-Figures}: This folder contains various plots that
  have been generated from simulations over the course of this project.
\item
  \textbf{Notes.txt}: This document is a ``progress journal'' of sorts,
  that contains my thoughts during development.
\end{enumerate}

Model Description

Model Assumptions

\begin{enumerate}
\def\labelenumi{\arabic{enumi}.}
\item
  No correlation between incompetency and ideology
\item
  There are three basic hiring modes: Default (D), Leadership
  Self-Replication (SR), Anti Leadership Self-Replication (ASR).

  \textbf{D Mode}: The meaning of default mode should be fairly
  self-evident. The leader has no bias in hiring, and will Therefore,
  the leader will select candidate with exactly the same bias admitted
  by the hiring pool.

  \textbf{SR Mode}: In self-replication mode, the leader is much more
  likely to hire a candidate that shares his worldview. How much more
  likely, then, is the question at hand. The most straightforward way to
  do this is to change the likelhoods associated with choosing each
  worldview (i.e, a different ``H\_config'' when in SR mode). But this
  requires actually modifying the hiring pool, which is both unrealistic
  and computationally expensive. The other way to do this is to set up
  ``interviews'', for which candidates are randomly chosen from the
  hiring pool, but the likleyhood that a candidate with each worldview
  is selected will be different. I've chosen P(Opp\_z) = .05, P(Opp\_nz)
  = .1, P(AB) = .3, P(Same) = .75, but these are slightly arbitrary. Is
  there a way to determine more realistic values? Notice that a possible
  strategy for mitigating polarization is designating multiple leaders
  with hiring power of different worldviews.

  \textbf{ASR Mode}: In anti self-replication mode, the leader is trying
  to maintain ideological diversity within the organization. We thus
  need a way to measure the polarization in the organization. The
  polarization metric will allow us to measure how well our
  anti-polarization ASR strategy is working. Ideally, we'd like to test
  different strategies to see which hiring strategy is most effective in
  mitigating polarization. In general, ASR hiring mode kicks in once
  polarization has exceeded a certain critical threshold, at which point
  hiring moves from default behavior to an anti-polarization strategy
  that, generally, hires individuals in such a way that will move the
  configuration of the organization towards a uniform distribution. The
  key questions here are:

  \textbf{Hiring Assumptions}:\\
  1. The leader can't distinguish between zealots and non-zealots. This
  means that if in SR mode, the leader is as likely to hire a non-zealot
  and zealot of the same worldview.

  \begin{enumerate}
  \def\labelenumii{\arabic{enumii}.}
  \setcounter{enumii}{1}
  \item
    \begin{enumerate}
    \def\labelenumiii{\arabic{enumiii}.}
    \setcounter{enumiii}{2}
    \tightlist
    \item
      The hiring pool is pre-filtered for competence.\\
    \item
      Hiring only takes place when someone has resigned or been fired.
      Thus the size of the organization stays constant.
    \end{enumerate}
  \end{enumerate}
\item
  Individuals change their mind through speaker-listener interactions.
  This is how the model is evolved.\\
\item
  Every individual within the organization has complete knowledge of the
  organization state. That is, everybody knows what everyone else thinks
  at any given point in time.
\end{enumerate}

Model Functionality

\begin{enumerate}
\def\labelenumi{\arabic{enumi}.}
\item
  \begin{longtable}[]{@{}lll@{}}
  \toprule
  Speaker & Listener & Final\tabularnewline
  \midrule
  \endhead
  A,A' & B & AB\tabularnewline
  A,A' & AB & A\tabularnewline
  B & A & AB\tabularnewline
  B & AB & B\tabularnewline
  \bottomrule
  \end{longtable}

  Note that this is modulated by preference falsification. Consider what
  happens when an individual is the listener in an interaction. We have
  various scenarios:

  \begin{enumerate}
  \def\labelenumii{\arabic{enumii}.}
  \tightlist
  \item
    If speaker is A and listener is B, then the listener is converted to
    AB
  \item
    If speaker is A' and listener is B, then the listener is subject to
    preference falsification, meaning that they may lie about being a B.
    That is, they will pretend to be closer to worldview A. We assume
    that this means that B is acting as an AB, and will therefore be
    converted to an A by the interaction.
  \item
    If speaker is B' and listener is A, then then, similarly, the
    listener will lie about being a true A and will pretend to be an AB
    in the interaction. They will subsequently be converted to a B.
  \item
    If speaker is A' and listener is A, then the listener will change to
    A' if the global state of the organization is such that the cost of
    becoming a zealot is sufficiently reduced.
  \end{enumerate}

  We specify scendario 4 further. Depending on how homogenous in A the
  organization is, A will turn to A'. It shouldn't be advantageous to
  switch until the organization is very homogenous in A. There is also a
  question about when it becomes socially unacceptable to not be a
  zealot. There is some interesting dynamics between people leaving
  because the organization is too homogenous and other people staying
  because there is social benefit to becoming a zealot, or, if the
  organization is extremely homogenous, social cost to not becoming one.

  We have a global scaling of probabilities with which A
  --\textgreater{} A' and B --\textgreater{} B' that is based on the \%
  of the organization that is either A or B. The idea here is that the
  more homogenous the organization, the less of a social cost there is
  for being a zealot; in fact, one may even be able to accrue social
  capital by becoming a zealot:

  \begin{enumerate}
  \def\labelenumii{\arabic{enumii}.}
  \tightlist
  \item
    Bias = \textless{}B\_1, B\_2, \ldots{}. ,B\_n\textgreater{}
    \textless{}==\textgreater{} Probs = \textless{}P\_1, P\_2, \ldots{}
    ,P\_n\textgreater{}
  \item
    This type of switch will only occur when the speaker is a zealot and
    the listener is a non-zealot with the same worldview
  \end{enumerate}

  The function mapping degree of homogeneity to probability of switching
  from non-zealot to zealot will be the same for both cases, A
  --\textgreater{} A' and B --\textgreater{} B'. How should this mapping
  behave? First, it seems reasonable that there would be a long leading
  tail. It will only become advantageous, either to accrue social
  capital or to avoid social destruction, to become a zealot if the
  organization is highly homogenous with respect to your worldview
  (\textgreater{}80\%?). People with high thresholds for homogeneity
  will likely end up as zealots if the organization tends towards
  homogeneity in their worldview.

  We have buckets \textless{}5, 10, 15, 20, 25, 30, 35, 40, 45, 50
  \ldots{} \textgreater{}. These are associated with the following
  probabilities: \textless{}.01, .02, .05, .07, .1, .135, .17, .205, .4,
  .45, .51, .58, .66, .75, .85, .95, .96, .97, .98. .99\textgreater{}

  Although I'd have like to use a continuous map here, for the sake of
  convenience, I've used discretized buckets.
\item
  Each individual has TOPP, or ``tolerance to opposition'' value. This
  is the percentage of the organization that must be of a competing
  worldview for an individual to resigns. TOPP is a measure of an
  individual's tolerance for being\\
  in the minority. This can be conceptualized, in psychometric terms, as
  how ``disagreeable'' an individual is. I've given TOPP a normal
  distribution in the organization's workforce. However, the question of
  which distribution is the right one to use should be revisited.
  Intuition suggests that perhaps a beta distribution with carefully
  selected parameters is the better option.
\item
  Any organization will have a natural steady turn over rate. We assume
  that individuals won't be fired for ideological reasons unless the
  leader is a zealot. In the normal case, once an individual leaves the
  organization (incompetence, surpassed threshold), they are replaced by
  someone from the hiring pool. Here are some factors to consider:

  \begin{enumerate}
  \def\labelenumii{\arabic{enumii}.}
  \tightlist
  \item
    The hiring pool might be ideologically biased --\textgreater{} note
    that unbiased hring (D mode) will select for this bias
  \item
    If in SR mode, then the bias of the leader will compound the hiring
    pool bias
  \item
    If in ASR mode, then the bias of the leader will counteract the
    hiring pool bias
  \end{enumerate}
\end{enumerate}

\end{document}
